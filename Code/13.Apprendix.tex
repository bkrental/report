\chapter{First Approach for System Architecture}

\begin{figure}[h]
\begin{center}
\includegraphics[scale = 0.3]{Images/mono_design.png}
\end{center}
\caption{Monolithic architecture design}
\end{figure}
In this approach, every module is packed into a tightly-coupled system. As the user message can be received in the form of text or voice, there are two ways to process input. If it is text-based, it will be transferred directly into a Natural Language Understanding module, which will classify the intents and entities of the sentence. On the other hand, voice-based will be processed by a Speech-to-Text module before into NLU. A Conversation Manager is in control of managing the flow of the conversation. Moreover, if the intent is a functional intent, an Answer Generator is used to find the answer. Lastly, a Machine Learning Pipeline module is used for creating an automated training pipeline.

The idea behind a Monolithic approach is straightforward. Since our system can be divided into 3 big modules: A Chatbot, A Machine Reading Comprehension module, and an MLOps module, this approach provides a clear and comprehensive view at a component level. Furthermore, compared to the next approach, it is better to keep the first approach for the purpose of better understanding the system. However, a Monolithic architecture has some big challenges for our system:
\begin{itemize}
    \item With this design, it also means that we will have many separate Machine Learning models (1 for the Intent classifier, 1 for Named Entity Recognizer, 1 for MRC), and 2 Retrievers (1 for FAQ and 1 for Document). As a result, the system is expensive and inefficient.
    \item It also suffers a scalability problem. With this many models, running the whole system is a problem. Therefore, it is almost impossible to scale this system.
    \item Due to its tightly-coupled nature, it is also difficult to integrate a CI/CD and CT pipeline. As the whole system must be loaded all at once, if we want to deploy a new model, it must restart the system.
\end{itemize}


\chapter{Data preparation team}
\begin{table}[ht]
    \centering
    \begin{tabular}{|l|c|}
        \hline
         \textbf{Full name} & \textbf{Student ID} \\
         \hline
         Phạm Văn Tâm & 2151254\\
         \hline
         Đỗ Nhật Quang & 2151246 \\
         \hline
         Lại Huy Anh & 2110725\\
         \hline
         Nguyễn Lê Hữu Quang & 2151247 \\
         \hline
         Nguyễn Hồ Phúc Nguyên & 2151236 \\
         \hline
         Lương Kim Long & 2151112 \\
         \hline
         Cao Minh Tuấn & 2151159 \\
         \hline
         Trần Hồng Lam & 2151218\\
         \hline
         Chu Thanh Đồng & 2151065 \\
         \hline
         Nguyễn Nam Vũ & 2151278 \\
         \hline
         Trần Trung Hiếu Nam & 2151122 \\
         \hline
         Trần Quang Nguyên Ánh & 2151047 \\
         \hline
         Lâm Hoàng Nhật & 2151238 \\
         \hline
         Đào Vũ Minh Trí & 2151153 \\
         \hline
         Lê Anh Tuấn & 2110643 \\
         \hline
         Nguyễn Hữu Đức & 2151189 \\
         \hline
         Trần Thị Minh Tâm & 2151032 \\
         \hline
         Nguyễn Giao Gia Bảo & 2110791 \\
         \hline
         Ngô Tiến Dũng & 2151185 \\
         \hline
         Vũ Huy Bảo & 2151177 \\
         \hline
         Đồng Trinh Hoàng Nguyên & 2151124 \\
         \hline
         Võ Hoàng Phúc & 2151029 \\
         \hline
         Văn Thiên Lâm & 2151110 \\
         \hline
         Nguyễn Gia Huy & 2151016 \\
         \hline
         Nguyễn Huy Hoàng & 2151197 \\
         \hline
         Phạm Minh Đạt & 2151007 \\
         \hline
         Phạm Gia Khang & 2151209 \\
         \hline
         Nguyễn Thái Bình Dương & 2051103 \\
         \hline
         Lâm Ngọc Cầm & 2151281 \\
         \hline
         Phạm Khắc Thanh Tùng & 2151274 \\
         \hline
         Nguyễn Thế Khang Hi & 2151070 \\
         \hline
         Võ Phú Thành & 2151259 \\
         \hline
         Huỳnh Lê Phi Long & 2151111 \\
         \hline
         Phạm Bình Dương & 2151283 \\
         \hline
         Nguyễn Xuân Khoa & 2051137 \\
         \hline
         Ngô Đình Anh Khoa & 2151212 \\
         \hline
         Trần Duy Trung Hiếu & 2151193 \\
         \hline
         Trần Quang Minh & 2151230 \\
         \hline
    \end{tabular}
    \caption{Data preparation team}
    \label{tab:my_label}
\end{table}